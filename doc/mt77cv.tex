
\begin{table}[h]
\begin{center}
\begin{tabular}{|p{0.5cm}|p{15.5cm}|}
\hline
P5 & Portabilidad a 32 y 64 bits. Formato de �ndices comprimido 
empleando c�digos $\Gamma$ de El�as.
Nueva mezcla de un �ndice en RAM con uno en disco, que resulta
m�s r�pida que mezclar dos �ndices en disco y agiliza operaci�n de
agregar un documento a un �ndice existente (opci�n \texttt{agregadoc} de
\texttt{operaindice}), as� como el indexado de colecciones. An�lisis de 
desempe�o (profiling) de indexado y
mezcla de indice en disco con �ndice en RAM.  El indexador puede
leer documentos PDF (ISO 32000-1, ver \cite{pdfiso}) y ahora opera
sobre colecciones, indexando por lotes en RAM de hasta 50MB y mezclando
con �ndice temporal en disco hasta completar todos los lotes. 
El buscador puede buscar cadenas, encontrando las palabras normalizadas
que aparezcan consecutivas (de 1 a 5 espacios de distancia).
El normalizador ahora incluye n�meros y puntos en algunos casos. 
Se indexa el t�tulo como palabra del documento. \\
\hline
P4 & Indexa documentos HTML. Formato de �ndices cambiado para disminuir 
tama�o e incluir fecha en relaci�n de documentos. Buscador
ordena resultados seg�n fecha (primero los m�s recientes) y
retorna resultados en JSON posibilitando paginaci�n.  \\
\hline
P3 & Divisi�n de �ndices en 2 archivos para agilizar actualizaci�n
distribuida.  Esquema de indexaci�n local cambiado para indexar
de a un documento en RAM y mezclar en disco --que aumenta l�mites
de indexaci�n para que dependan principalmente del espacio en disco y no en 
RAM.  Formato cambiado para lograr m�s velocidad
en b�squedas y disminuir tama�o de �ndices.  Permite nombres con espacios
y caracteres de espa�ol. Permite b�squedas con metainformaci�n sitio, 
tipo y titulo. Indexa documentos XML y ODF (.odt).
Comparaciones con Lucene y Amberfish.  \\
\hline
P2 & Mezclador opera enteramente en disco, aunque el conjunto de
posiciones lo mantiene en RAM (i.e la cantidad de documentos
retornados est� limitada por la cantidad de RAM --se han probado
m�ximo 10'000.000 con 1GB en RAM).  El indexador incluye URL origen del
documento. Permite b�squeda de varias palabras.
Normalizador elimina algunas palabras de espa�ol tan comunes que
no a�aden informaci�n a una consulta (art�culos, preposiciones, 
conjunciones).
\\
\hline
P1 & Indexador y mezclador operan enteramente en RAM 
(i.e a lo sumo indexan y mezclan en proporci�n a la RAM que tenga el 
servidor\footnote{Experimentando con 1GB en RAM hemos notado que un 
programa puede localizar alrededor de 256MB en un s�lo bloque.}),
en cuanto a formato de entrada s�lo soportan textos planos,
normalizan convirtiendo a may�sculas y eliminando n�meros, tildes y signos
de puntuaci�n.
\\
\hline
\end{tabular}
\caption{Control de versiones}
\label{buscadorcv}
\end{center}
\end{table}

